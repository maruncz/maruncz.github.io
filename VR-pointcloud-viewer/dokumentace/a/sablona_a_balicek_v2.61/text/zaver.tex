\chapter{Závěr}

Jako datový formát jsem vybral PLY, pro svou flexibilitu. V případě, že by se využila funkcionalita Point Cloud library by se využil i formát PCD.

Aplikace svůj demonstrační účel splnila. Lze načítat nekolik mračen bodů, pohybovat se v prostoru, měnit velkosti bodů a zobrazit skalární pole. Hardwarové nároky počítače jsou závislé pouze na složitosti načteného mračna.

Pro další vývoj aplikace bude vhodné přehodnotit celkový návrh a vybrat vhodnou knihovnu pro grafické rozhraní, aby bylo možné implementovat lepší uživatelské rozhraní, díky kterému bude možné pohodlné ovládání po přidání dalších funkcí a také získat uživatelskou zpětnou vazbu pro zkvalitnění 

Další vylepšování by se mohlo týkat výkonu. Pro virtuální realitu se musí vykreslovat vždy dva nezávislé snímky. Tyto snímky by mohly vykreslovat dvě grafické karty. V tomto by velmi pomohlo použití nových API jako je Vulkan nebo DirectX12, které jsou na tuto situaci lépe navrženy než předchozí API.
