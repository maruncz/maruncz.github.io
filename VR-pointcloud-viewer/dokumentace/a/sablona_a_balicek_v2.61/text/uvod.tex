\chapter*{Úvod}
\phantomsection
\addcontentsline{toc}{chapter}{Úvod}

Cílem práce je seznámit se s hardwarem i softwarem zařízení pro virtuální realitu, prověřit možnosti vývoje a navržení softwarového řešení pro zobrazování medicínských dat, jako například CT, MRI a RoScan, který je vyvíjen na VUT.

Práce byla zadána mým vedoucím a zaujala mě,  přestože jsem doposud o tomto tématu nic nevěděl a veškeré teoretické informace musím čerpat z internetu. 

V rámci mé práce jsou vstupní data ve formě  mračna bodů nebo meshe a úkolem této práce je zpracovat program, nebo alespoň základ, který bude schopen mračno bodů načíst a zobrazit v helmě virtuální reality.
Prvním úkolem bude vyhledání informací na internetu o technologiích virtuální reality, způsobech vývoje aplikací, datových formátech a vývoji 3D aplikace.

Poté musím vybrat vhodné knihovny a vhodný datový formát tak, aby má aplikace mohla fungovat s mě dostupným headsetem HTC Vive a aby byl datový formát kompatibilní s již používanými programy.

Díky tomuto programu bude moct uživatel vidět zobrazené data jako plastický obraz, což v medicíně může umožnit zkvalitnění diagnostiky. 