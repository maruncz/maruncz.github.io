%% Název práce:
%  První parametr je název v originálním jazyce,
%  druhý je překlad v angličtině nebo češtině (pokud je originální jazyk angličtina)
\nazev{Programový modul pro zobrazení mračna bodů ve virtuální realitě}{Software Module for Point Clouds Display in Virtual Reality}

%% Jméno a příjmení autora ve tvaru
%  [tituly před jménem]{Křestní}{Příjmení}[tituly za jménem]
\autor[]{Martin}{Bařinka}

%% Jméno a příjmení vedoucího včetně titulů
%  [tituly před jménem]{Křestní}{Příjmení}[tituly za jménem]
% Pokud vedoucí nemá titul za jménem, smažte celý řetězec '[...]'
\vedouci[prof.\ Ing.]{Luděk}{Žalud}[Ph.D.]

%% Jméno a příjmení oponenta včetně titulů
%  [tituly před jménem]{Křestní}{Příjmení}[tituly za jménem]
% Pokud nemá titul za jménem, smažte celý řetězec '[...]'
% Uplatní se pouze v prezentaci k obhajobě
%\oponent{}{}

%% Označení oboru studia
% První parametr je obor v originálním jazyce,
% druhý parametr je překlad v angličtině nebo češtině
%\oborstudia{Teleinformatika}{Teleinformatics}

%% Označení ústavu
% První parametr je název ústavu v originálním jazyce,
% druhý parametr je překlad v angličtině nebo češtině
\ustav{Ústav automatizace a měřicí techniky}{Department of Control and Instrumentation} 

%% Rok obhajoby
\rok{Rok}
\datum{10.\,1.\,2017} % Uplatní se pouze v prezentaci k obhajobě

%% Místo obhajoby
% Na titulních stránkách bude automaticky vysázeno VELKÝMI písmeny
\misto{Brno}

%% Abstrakt
\abstrakt{Cílem práce bylo seznámit se s hardwarem a softwarem zařízení pro virtuální realitu, prozkoumat možnosti vývoje a navržením softwarového řešení pro zobrazování medicínských dat, z CT, MRI nebo RoScan vyvíjeným na VUT. }{The goal of my task was to understand hardware and software of equipment for virtual reality, investigate possibilities to develop and design software solution for medical dates from CT, MRI or RoScan, developed at VUT.}

%% Klíčová slova
\klicovaslova{Virtuální realita; mračna bodů; mesh; 3D zobrazování}%
	{Virtual reality; pointcloud; mesh; 3D rendering}

%% Poděkování
\podekovanitext{Rád bych poděkoval vedoucímu diplomové práce panu prof. Ing. Luděk Žalud Ph.D.\ za odborné vedení, konzultace, trpělivost a podnětné návrhy k~práci.}